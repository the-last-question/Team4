\chapter{Introdução}
\label{chap:intro}

A odometria \'e uma t\'ecnica usada para medir a dist\^ancia percorrida. Sendo de grande import\^ancia no ramo da Rob\'otica, pois \'e imprescind\'ivel que um rob\^o m\'ovel consiga se locomover de maneira a alcan\c{c}ar o seu ponto de destino.
Para tal, \'e necess\'ario que este rob\^o possua uma forma de se localizar no espa\c{c}o, atrav\'es de algum sensor que lhe d\^e esta informa\c{c}\~ao.
\'E importante tmb\'em que ele possa evitar situa\c{c}\~oes perigosas, como colis\~oes, buracos e locais com condi\c{c}\~oes clim\'aticas prejudicias ao rob\^o. De forma, a alcan\c{c}ar o local desejado.
Al\'em disso, ele deve registrar ou j\'a ter registrado o mapa do local em que se encontra e ser capaz de interpretar esta informa\c{c}\~ao para ser utilizada ao seu favor, fazendo com que o mesmo consiga se deslocar no ambiente de forma segura at\'e encontrar a sua meta.

%--------- NEW SECTION ----------------------
\section{Objetivos}
\label{sec:obj}

Estudar o funcionamento da tecnologia Pozyx para aferi\c{c}\~ao do posicionamento de um rob\^o m\'ovel, de forma a auxiliar a odometria do mesmo.


\subsection{Objetivos Específicos}
\label{ssec:objesp}

Estudar funcionamento do Pozyx;
Descrever a forma de utiliza\c{c}\~ao do posicionamento para odometria do ro\^o;
Pesquisar as suas diferen\c{c}as para outras tecnologias de aux\'ilio a odometria;
Buscar as aplica\c{c}\~oes do Pozyx na rob\'otica.


%--------- NEW SECTION ----------------------
\section{Justificativa}
\label{sec:justi}

O Pozyx \'e uma solu\c{c} de hardware e software que fornece informa\c{c}\~oes precisas de posicionamento e movimento com boa precis\~ao.
Esta precis\~ao s\'o p\^ode ser alcan\c{c}ada devido a sua tecnologia de banda ultralarga, algoritmos inteligentes e aprendizado da m\'aquina.
Logo, o estudo da utiliza\c{c}\~ao do Pozyx \'e imprescind\'ivel para t\^e-lo como op\c{c}\~ao na aferi\c{c}\~ao do posicionamento do ro\^o para a odometria adequada do mesmo. 


%--------- NEW SECTION ----------------------
\section{Organização do documento}
\label{section:organizacao}

Este documento apresenta $5$ capítulos e está estruturado da seguinte forma:

\begin{itemize}

  \item \textbf{Capítulo \ref{chap:intro} - Introdução}: Contextualiza o âmbito, no qual a pesquisa proposta está inserida. Apresenta, portanto, a definição do problema, objetivos e justificativas da pesquisa e como este \thetypeworkthree está estruturado;
  \item \textbf{Capítulo \ref{chap:fundteor} - Fundamentação Teórica}: Descreve o funcionamento do Pozyx e aborda as diferenças das outras tecnologias;
  \item \textbf{Capítulo \ref{chap:mat} - Materiais e Métodos}: ;
  \item \textbf{Capítulo \ref{chap:result} - Resultados}: ;
  \item \textbf{Capítulo \ref{chap:conc} - Conclusão}: Apresenta as conclusóes, contribuições e algumas sugestões de atividades de pesquisa a serem desenvolvidas no futuro.

\end{itemize}
