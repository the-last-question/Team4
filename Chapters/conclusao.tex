\chapter{Conclusão}
\label{chap:conc}

O sistema de localização de corpos em ambientes indoor requer uma boa precisão na estimativa de pequenas distâncias, assim como ferramentas capazes de diferenciar freq. Desse modo, conforme apresentado a tecnologia Pozyx ao utilizar a UWB apresenta um bom desenvolvimento, uma vez que é menos susceptíveis a interferências por sinais refletidos e posicionar o robô de forma adequada, além de apresentar uma eficiência energética melhor do que as demais tecnologias apresentadas no mercado. Ademais, possui também como vantagem uma implementação, apesar de robusta, de fácil entendimento e monitoramento e manutenção remotas. Inclusive, apresenta melhor performance quando está relacionada com outras ferramentas de diminuição de erros, como os filtros. Sendo assim, essa ferramenta de localização têm um elevado potencial de desenvolvimento futuro, em diversas aplicações, inclusive em ambientes outdoors, a partir do momento que consiga atingir um alcance maior de transmissão de sinais e, com isso, baratear os custos da aplicação desta tecnologia.

% \section{Considerações finais}
% \label{sec:consid}



